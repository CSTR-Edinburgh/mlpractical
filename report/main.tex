%% Template for MLP Coursework 2 / 6 November 2017 

%% Based on  LaTeX template for ICML 2017 - example_paper.tex at 
%%  https://2017.icml.cc/Conferences/2017/StyleAuthorInstructions

\documentclass{article}
\usepackage[T1]{fontenc}
\usepackage{amssymb,amsmath}
\usepackage{txfonts}
\usepackage{microtype}

% For figures
\usepackage{graphicx}
\usepackage{subfigure} 

% For citations
\usepackage{natbib}

% For algorithms
\usepackage{algorithm}
\usepackage{algorithmic}

% the hyperref package is used to produce hyperlinks in the
% resulting PDF.  If this breaks your system, please commend out the
% following usepackage line and replace \usepackage{mlp2017} with
% \usepackage[nohyperref]{mlp2017} below.
\usepackage{hyperref}
\usepackage{url}
\urlstyle{same}

% Packages hyperref and algorithmic misbehave sometimes.  We can fix
% this with the following command.
\newcommand{\theHalgorithm}{\arabic{algorithm}}


% Set up MLP coursework style (based on ICML style)
\usepackage{mlp2019}
\mlptitlerunning{MLP Coursework 1 (\studentNumber)}
\bibliographystyle{icml2017}


\DeclareMathOperator{\softmax}{softmax}
\DeclareMathOperator{\sigmoid}{sigmoid}
\DeclareMathOperator{\sgn}{sgn}
\DeclareMathOperator{\relu}{relu}
\DeclareMathOperator{\lrelu}{lrelu}
\DeclareMathOperator{\elu}{elu}
\DeclareMathOperator{\selu}{selu}
\DeclareMathOperator{\maxout}{maxout}


%% You probably do not need to change anything above this comment

%% REPLACE this with your student number
\def\studentNumber{sXXXXXXX}

\begin{document} 

\twocolumn[
\mlptitle{MLP Coursework 2: Investigating the Implementation and Optimization of Convolutional Networks}

\centerline{\studentNumber}

\vskip 7mm
]

\begin{abstract} 
The abstract should be a few sentences (100--200 words) long,  providing a concise summary of the contents of your report including the key research question(s) addressed, the methods explored, the data used, and the findings of the experiments.
\end{abstract} 

\section{Introduction}
\label{sec:intro}
This document provides a template for the MLP coursework 2 report.  This template structures the report into sections, which you are recommended to use, but can change if you wish.  If you want to use subsections within a section that is fine, but please do not use any deeper structuring. In this template the text in each section will include an outline of what you should include in each section, along with some practical LaTeX examples (for example figures, tables, algorithms).  Your document should be no longer than \textbf{six pages},  with an additional page (or more!) allowed for references.

The introduction should place your work in context, giving the overall \textbf{motivation} for the work, and clearly outlining the \textbf{objectives} of the work and the \textbf{research questions} you have explored -- in this case the implementation of convolutional network, debugging of optimization problems and generalization improvement methods.  Most of the report will be to do with part 2.  



\section{Implementing convolutional networks} 
Concisely explain the idea of convolutional layers and explain what you did in the implementation.  There is no need to include chunks of code but sufficiently detailed description that enables someone else to re-implement and reproduce your results. You should also discuss the pros and cons of different approaches to implementing convolutional layers, in terms of computational efficiency (running time, storage demands), and any analysis of your own implementation.

\section{Optimization problems in convolutional networks}
This section should introduce the main part of the report which is to do with debugging optimization problems in deep neural networks, and ways of improving such problems. State clearly what you think the actual problem was, that caused the issues, as well as what causes it. Then, begin to present your motivations behind the methods you used to improve the problem. This section should present, in your own words, the different approaches you have adopted, explaining how they work and what the key differences between them are.  You may reference the literature where appropriate.   You should also outline the key research questions that you have addressed in this work.

If you present algorithms, you can use the \verb+algorithm+ and \verb+algorithmic+ environments to format pseudocode (for instance, Algorithm~\ref{alg:example}). These require the corresponding style files, \verb+algorithm.sty+ and \verb+algorithmic.sty+ which are supplied with this package. 

\begin{algorithm}[ht]
\begin{algorithmic}
   \STATE {\bfseries Input:} data $x_i$, size $m$
   \REPEAT
   \STATE Initialize $noChange = true$.
   \FOR{$i=1$ {\bfseries to} $m-1$}
   \IF{$x_i > x_{i+1}$} 
   \STATE Swap $x_i$ and $x_{i+1}$
   \STATE $noChange = false$
   \ENDIF
   \ENDFOR
   \UNTIL{$noChange$ is $true$}
\end{algorithmic}
  \caption{Bubble Sort}
  \label{alg:example}
\end{algorithm}

\section{Improving the generalization performance of convolutional networks}
This section should introduce what generalization is, how it relates to overfitting and why it is important. Then you should describe the methods you applied, as well as why, in your attempts to improve the generalization performance of your model. This section should present, in your own words, the different approaches you have adopted, explaining how they work and what the key differences between them are.  You may reference the literature where appropriate.   You should also outline the key research questions that you have addressed in this work.

\section{Experiments}
This section should include a concise description of the CIFAR100 task and  data -- be precise: for example state the size of the training, validation, and test sets.  
This section should cover the experiments carried out. For each experiment, make clear why it was carried out, what you were trying to discover. Describe carefully how you carried out the experiments, mentioning and justifying any hyperparameter settings.  As always, your aim is to give enough information so that someone else (e.g. another MLP group) could reproduce the experiment precisely.  Note that it is interesting to consider both accuracy / generalisation and runtime / memory requirements.

Present the experimental results clearly and concisely.  Usually a result is in comparison or contrast to a result from another approach please make sure that these comparisons/contrasts are clearly presented.  You can facilitate comparisons either using graphs with multiple curves or (if appropriate, e.g. for accuracies) a results table. You need to avoid having too many figures, poorly labelled graphs, and graphs which should be comparable but which use different axis scales. A good presentation will enable the reader to compare trends in the same graph -- each graph should summarise the results relating to a particular research (sub)question.

There is no need to include code or specific details about the compute environment.

As before, your experimental sections should include graphs (for instance, figure~\ref{fig:sample-graph}) and/or tables (for instance, table~\ref{tab:sample-table})\footnote{These examples were taken from the ICML template paper.}, using the \verb+figure+ and \verb+table+ environments, in which you use \verb+\includegraphics+ to include an image (pdf, png, or jpg formats).  Please export graphs as 
\href{https://en.wikipedia.org/wiki/Vector_graphics}{vector graphics}
rather than \href{https://en.wikipedia.org/wiki/Raster_graphics}{raster
files} as this will make sure all detail in the plot is visible.
Matplotlib supports saving high quality figures in a wide range of
common image formats using the
\href{http://matplotlib.org/api/pyplot_api.html\#matplotlib.pyplot.savefig}{\texttt{savefig}}
function. \textbf{You should use \texttt{savefig} rather than copying
the screen-resolution raster images outputted in the notebook.} An
example of using \texttt{savefig} to save a figure as a PDF file (which
can be included as graphics in a \LaTeX document is given in the coursework document.

If you need a figure or table to stretch across two columns use the \verb+figure*+ or \verb+table*+ environment instead of the \verb+figure+ or \verb+table+ environment.  Use the \verb+subfigure+ environment if you want to include multiple graphics in a single figure.

\begin{figure}[tb]
\vskip 5mm
\begin{center}
\centerline{\includegraphics[width=\columnwidth]{icml_numpapers}}
\caption{Historical locations and number of accepted papers for International
  Machine Learning Conferences (ICML 1993 -- ICML 2008) and
  International Workshops on Machine Learning (ML 1988 -- ML
  1992). At the time this figure was produced, the number of
  accepted papers for ICML 2008 was unknown and instead estimated.}
\label{fig:sample-graph}
\end{center}
\vskip -5mm
\end{figure} 

\begin{table}[tb]
\vskip 3mm
\begin{center}
\begin{small}
\begin{sc}
\begin{tabular}{lcccr}
\hline
\abovespace\belowspace
Data set & Naive & Flexible & Better? \\
\hline
\abovespace
Breast    & 95.9$\pm$ 0.2& 96.7$\pm$ 0.2& $\surd$ \\
Cleveland & 83.3$\pm$ 0.6& 80.0$\pm$ 0.6& $\times$\\
Glass2    & 61.9$\pm$ 1.4& 83.8$\pm$ 0.7& $\surd$ \\
Credit    & 74.8$\pm$ 0.5& 78.3$\pm$ 0.6&         \\
Horse     & 73.3$\pm$ 0.9& 69.7$\pm$ 1.0& $\times$\\
Meta      & 67.1$\pm$ 0.6& 76.5$\pm$ 0.5& $\surd$ \\
Pima      & 75.1$\pm$ 0.6& 73.9$\pm$ 0.5&         \\
\belowspace
Vehicle   & 44.9$\pm$ 0.6& 61.5$\pm$ 0.4& $\surd$ \\
\hline
\end{tabular}
\end{sc}
\end{small}
\caption{Classification accuracies for naive Bayes and flexible 
Bayes on various data sets.}
\label{tab:sample-table}
\end{center}
\vskip -3mm
\end{table}






\section{Discussion}
Your discussion should interpret the results, both in terms of summarising the outcomes of a particular experiment, and attempting to relate to the research question(s) which motivated the experiments . A good report would have some analysis, resulting in an understanding of why particular results are observed, perhaps with reference to the literature. Use bibtex to organise your references -- in this case the references are in the file \verb+example-refs.bib+.  Here is a an example reference \citep{langley00}.  

A good report will relate the results to  published work which can help to give a better understanding of your work -- related approaches, other work on the same data, ideas for future work. 



% \textbf{Conclusions:}  


\section{Conclusions}
\label{sec:concl}
The conclusions section should concisely summarise what you have learned from the experiments you carried out, and relate the findings of your work to the  research questions you posed at the start.   It is good if the conclusion from one experiment influenced what you did in later experiments -- your aim is to learn from your experiments.   

A good conclusions section would also include a further work discussion, building on work done so far, and referencing the literature where appropriate.

\bibliography{example-refs}

\end{document} 

